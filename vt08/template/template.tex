\documentclass[a4paper,10pt,twoside]{article}

\usepackage[inner=3cm,top=3cm,outer=2cm,bottom=3cm]{geometry}
\usepackage[T1]{fontenc}
\usepackage{amssymb}
\usepackage{fancyhdr}
\usepackage{fancyvrb}
\usepackage{graphicx}
\usepackage{xcolor}
\usepackage{listings}
\ifxetex
	% i xetex behövs inga fulhack för åäö
\else
	% åäö-hax för icke-xelatex
	\usepackage[utf8]{inputenc}
	% Fixar så man kan ha åäö i kodkommentarer
	\lstset{literate={ö}{{\"o}}1
		{ä}{{\"a}}1
		{å}{{\aa}}1
		{Ö}{{\"O}}1
		{Ä}{{\"A}}1
		{Å}{{\AA}}1
	}
\fi

\definecolor{dark-blue}{rgb}{0, 0, 0.6}
\definecolor{javared}{rgb}{0.6,0,0} % for strings
\definecolor{javagreen}{rgb}{0.25,0.5,0.35} % comments
\definecolor{javapurple}{rgb}{0.5,0,0.35} % keywords
\definecolor{javadocblue}{rgb}{0.25,0.35,0.75} % javadoc

\lstset{language=Java,
	basicstyle=\ttfamily,
	% Här anges alltså vilka färger som sak användas.
	% Vill ni inte ha färger, kommentera ut nästföljande fyra rader.
	keywordstyle=\color{javapurple}\bfseries,
	stringstyle=\color{javared},
	commentstyle=\color{javagreen},
	morecomment=[s][\color{javadocblue}]{/**}{*/},
	numbers=left, % Fixar radnumrering i vänstermarginalen
	numberstyle=\footnotesize,
	title=\lstname,
	showstringspaces=false,
	fancyvrb=true,
	extendedchars=true,
	breaklines=true,
	breakatwhitespace=true,
	tabsize=4 %Indenteringsstorlek
}

% Header och footer
\pagestyle{fancy}\headheight 13pt
%\lhead{\course\ -\ Inlämning \homeworknumber}
%\rhead{\theauthor\ -\ \thedate}
\fancyfoot{}
\fancyfoot[LE,RO]{\thepage}
\date{\thedate}
\author{\theauthor}
\renewcommand{\headrulewidth}{0pt}
\renewcommand{\footrulewidth}{0pt}

\def\theauthor{Hampus Fristedt} % TODO: stoppa in ditt namn här
\def\coursename{Introduktion till datalogi}
\def\course{DD1339}
\def\groupnumber{1} % TODO: stoppa in gruppnummer här
\def\courseassistant{Peter Boström} % TODO: stoppa in gruppassistent här


% Här börjar inlämningsspecifikt
\def\homeworknumber{VT8} % TODO: stoppa in vilken hemläxa det är här
\def\thedate{\today} % Byt ut om du vill ha annat datum på inlämningen

\title{Inlämning \homeworknumber\ - \course\ \coursename}

\begin{document}

% Labbcover + clearpage för blank baksida bakom den
\input{labbcover.tex}
\clearpage
\thispagestyle{empty}
\mbox{} % empty page for duplex
\clearpage 

% Huvudinlämning
\setcounter{page}{1}
\section{Buggar}
\lstinputlisting{../bug1.go}
\lstinputlisting{../bug2.go}

\section{Producers and consumers}
\subsection{Vad händer om man byter plats på satserna wgp.Wait() och close(ch) i slutet av main-funktionen?}
Om vi stänger kanalen innan vi är säkra på att producenterna är klara finns det en stark risk att producenterna försöker skicka data till en stäng kanal, vilket ger fel i runtime.

\subsection{Vad händer om man flyttar close(ch) från main-funktionen och i stället stänger kanalen i slutet av funktionen Produce?}
Då kommer kanalen stängas när den första gorutinen som kör Produce är klar, vilket förstör för alla andra producenter som försöker skicka till kanalen.

\subsection{Vad händer om man tar bort satsen close(ch) helt och hållet?}
Då kommer gorutinerna som kör Consume att köra till programet avslutas eftersom kanalen aldrig stängs och Consume loopar tills kanalen stängs.

\subsection{Vad händer om man ökar antalet konsumenter från 2 till 4?}
Då paralliserar vi konsumerandet ytterligare och programmet blir snabbare.

\subsection{Kan man vara säker på att alla strängar blir utskrivna innan programmet stannar?}
Nej, eftersom programet avslutas så fort producenterna är klara, kanalen är stängd, och tiden har printats. Det är ingenting som garanterar att alla konsumenter blir klara.
\lstinputlisting{../many2many.go}

\section{Oracle}
\lstinputlisting{../oracle.go}
\end{document}

